\documentclass[a4paper,11pt]{article}
\title{A Small Collection Of Mathematical Formulas For The 
Prospective Statistician}
\author{Joshua Simon  \\
	Otto-Friedrich-University Bamberg \\
	}

\date{\today}

% Packages.
\usepackage{amssymb}
\usepackage{amsmath}


\begin{document}

\maketitle


%\begin{abstract}
%Short introduction to subject of the paper \ldots 
%\end{abstract}

\section{Arithmetic}
\paragraph{Logarithms.} In the following let $a$, $b > 0$ and $n \in \mathbb{R}$. Then
\begin{gather*}
    ln(a \cdot b) = ln(a) + ln(b) \\
    ln(\frac{a}{b}) = ln(a) - ln(b) \\
    ln(a^n) = n \cdot ln(a) \\
    e^{ln(a)} = ln(e^{a}) = a.
\end{gather*}

\paragraph{Binomial Coefficient.} Suppose $0 < k \leq n$. Then
\begin{gather*}
    \binom{n}{k} = \frac{n!}{k! \cdot (n-k)!}
\end{gather*}
where $n! = 1 \cdot 2 \cdot \ldots \cdot n$.


%%%%%%%%%%%%%%%%%%%%%%%%%%%%%%%%%%%%%%%%%%%%%%%%%%%%%%%%%%%%%%

\section{Differential Calculus} \label{DifferentialCalculus}
\paragraph{Elementary Derivatives.} In the following let $f$ be a real-valued function of $x$ and $a, b, r \in \mathbb{R}$. Then the derivative of $f$ is given by $f'$ as

\begin{align*}
    f(x) = x^r &\rightarrow f'(x) = r \cdot x^{r - 1} 
    &f(x) = e^x &\rightarrow f'(x) = e^x \\
    f(x) = r^x &\rightarrow f'(x) = r^x \cdot ln(r) 
    &f(x) = e^{a \cdot x + b} &\rightarrow f'(x) = a \cdot e^{a \cdot x + b} \\
    f(x) = \frac{1}{x^r}  &\rightarrow f'(x) = - r \cdot x^{-r - 1}
    &f(x) = sin(x) &\rightarrow f'(x) = cos(x) \\
    f(x) = ln(x) &\rightarrow f'(x) = \frac{1}{x} 
    &f(x) = cos(x) &\rightarrow f'(x) = - sin(x).
\end{align*}

\paragraph{Differentiation rules.} In the following let $h, u, v$ be real-valued functions of $x$.
\begin{itemize}
    \item Product Rule: $h(x) = u(x) \cdot v(x) \rightarrow h'(x) = u'(x) \cdot v(x) + u(x) \cdot v'(x)$
    \item Chain Rule: $h(x) = u(v(x)) \rightarrow  h'(x) = u'(v(x)) \cdot v'(x)$
    \item Reciprocal Rule: $h(x) = \frac{1}{v(x)} \rightarrow  h'(x) = \frac{v'(x)}{(v(x))^2}$
    \item Quotient Rule: $h(x) = \frac{u(x)}{v(x)} \rightarrow  h'(x) = \frac{v(x)u'(x) - u(x)v'(x)}{(v(x))^2}$
\end{itemize}

\paragraph{Gradient.} Let $f$ be a differentiable fuction of $n$ variables. Then
\begin{gather*}
    \mathbf{grad}(f) = \nabla f = \left( \frac{\partial}{\partial x_1} f(x) \dots \frac{\partial}{\partial x_n} f(x) \right)^T.
\end{gather*}

\paragraph{Jacobian Matrix.} Let $f: \mathbb{R}^{n} \rightarrow \mathbb{R}^{m}$ be a differentiable function of $n$ variables. Then the Jacobian matrix $\mathbf {J}$ of $f$ is an $m \times n$ matrix whose $(i,j)$th entry is ${\mathbf {J}_{ij}={\frac {\partial f_{i}}{\partial x_{j}}}}$. Which is 
\begin{gather*}
    {\mathbf {J} ={\begin{bmatrix}{\dfrac {\partial f }{\partial x_{1}}}&\cdots &{\dfrac {\partial f }{\partial x_{n}}}\end{bmatrix}}={\begin{bmatrix}\nabla ^{T}f_{1}\\\vdots \\\nabla ^{T}f_{m}\end{bmatrix}}={\begin{bmatrix}{\dfrac {\partial f_{1}}{\partial x_{1}}}&\cdots &{\dfrac {\partial f_{1}}{\partial x_{n}}}\\\vdots &\ddots &\vdots \\{\dfrac {\partial f_{m}}{\partial x_{1}}}&\cdots &{\dfrac {\partial f_{m}}{\partial x_{n}}}\end{bmatrix}}}
\end{gather*}
where ${\nabla ^{T}f_{i}}$ is the transpose (row vector) of the gradient of the $i$th component.

\paragraph{Symmetry Of Second Derivatives (Schwarz's theorem).} The order of taking partial derivatives of a function $f(x) = f(x_1, x_2, \dots , x_n)$ of $n$ variables is interchangeable
\begin{gather*}
    \frac{\partial}{\partial x_i} \left( \frac{\partial}{\partial x_j} f(x) \right) = \frac{\partial}{\partial x_j} \left( \frac{\partial}{\partial x_i} f(x) \right).
\end{gather*}


%%%%%%%%%%%%%%%%%%%%%%%%%%%%%%%%%%%%%%%%%%%%%%%%%%%%%%%%%%%%%%


\section{Integral Calculus} \label{integralCalculus}

\paragraph{Fundamental Theorem Of Calculus.} 
Let  $f$ be a real-valued function on a closed interval $[a,b]$ and $F$ an antiderivative of $f$ in $[a,b]$: $F'(x)=f(x)$. If $f$ is Riemann integrable on $[a,b]$ then
\begin{gather*}
    \int _{a}^{b}f(x)\,dx=F(b)-F(a).
\end{gather*}

\paragraph{Elementary Integrals.}
\begin{align*}
    \int x^r dx &= \frac{x^{r + 1}}{r+1} + C \text{ with } (r \neq -1) 
    &\int e^x dx &= e^x + C \\
    \int r^x dx &= \frac{r^{x}}{ln(r)} + C \text{ with } (r > 1, r \neq 1) 
    &\int e^{a \cdot x + b} dx &= \frac{e^{a \cdot x + b}}{a} + C \\
    \int \frac{1}{x} dx &= ln(x) + C 
    &\int ln(x) dx &= x \cdot (ln(x) - 1) + C \\
    \int \frac{1}{x^2} dx &= - \frac{1}{x} + C 
    &\int sin(x) dx &= - cos(x) + C \\
    \int \frac{1}{x^3} dx &= - \frac{1}{2x^2} + C
    &\int cos(x) dx &= sin(x) + C
\end{align*}

\paragraph{Integration By Parts.} Let $u$ and $v$ be two continuously differentiable functions of $x$ in $[a,b]$. Then
\begin{gather*}
   \int_{a}^{b} u(x)v'(x) dx = \left[ u(x)v(x) \right]_a^b - \int_{a}^{b} u'(x)v(x) dx.
\end{gather*}

\paragraph{Integration By Substitution.}



%%%%%%%%%%%%%%%%%%%%%%%%%%%%%%%%%%%%%%%%%%%%%%%%%%%%%%%%%%%%%%


\begin{thebibliography}{9}
    \bibitem[Bron]{Bron} \emph{Taschenbuch der Mathematik.}
    I. N. Bronstein, K. A. Semendjajew, G. Musiol, H. Mühlig, 11. Auflage, 2020.
\end{thebibliography}

\end{document}